
\section{Let's Implement λ-Calculus}
\subsection{}

\frame{\frametitle{\color{white} dummy}
  \begin{itemize}
    \item {\color{lightgray} What can Haskell do?}
    \item {\color{lightgray} What is λ-Calculus?}
    \item {\color{lightgray} Why Implement λ-Calculus?}
    \item {Let's Implement λ-Calculus}
    \item {\color{lightgray} Questions?}
    \item {\color{lightgray} References}
  \end{itemize}
}

\frame{\frametitle{First Expression and evaluate (\jref{https://github.com/godfat/sandbox/blob/master/haskell/fpug/01/00.lhs
}{source})}
\footnotesize
\usebeamerfont*{font DejaVu Sans Mono}
module Main where \\
\color{white}. \\
data Expression = Literal Integer \\
 \ \ \ \ \ \ \ \ \ \ \ \ \ \ \ \ | Plus Expression Expression \\
{\color{white}.} \\
evaluate :: Expression -> Integer \\
evaluate (Literal i) \ \ \ \ \ \ \ = i \\
evaluate (Plus expr0 expr1) = \\
 \ \ evaluate expr0 + evaluate expr1 \\
{\color{white}.} \\
test0 = evaluate (Literal 1) \\
test1 = evaluate (Plus (Literal 1) (Literal 2)) \\
test2 = evaluate (Plus (Plus (Literal 1) \\
 \ \ \ \ \ \ \ \ \ \ \ \ \ \ \ \ \ \ \ \ \ \ \ \ \ \ \ \ \ (Literal 2)) \\
 \ \ \ \ \ \ \ \ \ \ \ \ \ \ \ \ \ \ \ \ \ \ \ (Literal 3))
}

\frame{\frametitle{First Expression and evaluate (\jref{https://github.com/godfat/sandbox/blob/master/haskell/fpug/01/00.lhs
}{source})}
\footnotesize
\usebeamerfont*{font DejaVu Sans Mono}
module Main where \\
{\color{white}.} \\
data Expression = Literal Integer \\
 \ \ \ \ \ \ \ \ \ \ \ \ \ \ \ \ | Plus Expression Expression \\
\color{white}. \\
evaluate :: Expression -> Integer \\
evaluate (Literal i) \ \ \ \ \ \ \ = i \\
evaluate (Plus expr0 expr1) = \\
 \ \ evaluate expr0 + evaluate expr1 \\
{\color{white}.} \\
test0 = evaluate (Literal 1) \\
test1 = evaluate (Plus (Literal 1) (Literal 2)) \\
test2 = evaluate (Plus (Plus (Literal 1) \\
 \ \ \ \ \ \ \ \ \ \ \ \ \ \ \ \ \ \ \ \ \ \ \ \ \ \ \ \ \ (Literal 2)) \\
 \ \ \ \ \ \ \ \ \ \ \ \ \ \ \ \ \ \ \ \ \ \ \ (Literal 3))
}

\frame{\frametitle{First Expression and evaluate (\jref{https://github.com/godfat/sandbox/blob/master/haskell/fpug/01/00.lhs
}{source})}
\footnotesize
\usebeamerfont*{font DejaVu Sans Mono}
module Main where \\
{\color{white}.} \\
data Expression = Literal Integer \\
 \ \ \ \ \ \ \ \ \ \ \ \ \ \ \ \ | Plus Expression Expression \\
{\color{white}.} \\
evaluate :: Expression -> Integer \\
evaluate (Literal i) \ \ \ \ \ \ \ = i \\
evaluate (Plus expr0 expr1) = \\
 \ \ evaluate expr0 + evaluate expr1 \\
\color{white}. \\
test0 = evaluate (Literal 1) \\
test1 = evaluate (Plus (Literal 1) (Literal 2)) \\
test2 = evaluate (Plus (Plus (Literal 1) \\
 \ \ \ \ \ \ \ \ \ \ \ \ \ \ \ \ \ \ \ \ \ \ \ \ \ \ \ \ \ (Literal 2)) \\
 \ \ \ \ \ \ \ \ \ \ \ \ \ \ \ \ \ \ \ \ \ \ \ (Literal 3))
}

\frame{\frametitle{First Expression and evaluate (\jref{https://github.com/godfat/sandbox/blob/master/haskell/fpug/01/00.lhs
}{source})}
\footnotesize
\usebeamerfont*{font DejaVu Sans Mono}
module Main where \\
{\color{white}.} \\
data Expression = Literal Integer \\
 \ \ \ \ \ \ \ \ \ \ \ \ \ \ \ \ | Plus Expression Expression \\
{\color{white}.} \\
evaluate :: Expression -> Integer \\
evaluate (Literal i) \ \ \ \ \ \ \ = i \\
evaluate (Plus expr0 expr1) = \\
 \ \ evaluate expr0 + evaluate expr1 \\
{\color{white}.} \\
test0 = evaluate (Literal 1) \\
test1 = evaluate (Plus (Literal 1) (Literal 2)) \\
test2 = evaluate (Plus (Plus (Literal 1) \\
 \ \ \ \ \ \ \ \ \ \ \ \ \ \ \ \ \ \ \ \ \ \ \ \ \ \ \ \ \ (Literal 2)) \\
 \ \ \ \ \ \ \ \ \ \ \ \ \ \ \ \ \ \ \ \ \ \ \ (Literal 3))
}

\frame{\frametitle{Algebraic Datatypes}
\footnotesize
\usebeamerfont*{font DejaVu Sans Mono}
{\color{white} module Main where \\
.} \\
data Expression = Literal Integer \\
 \ \ \ \ \ \ \ \ \ \ \ \ \ \ \ \ | Plus Expression Expression \\
{\color{white}. \\
evaluate :: Expression -> Integer \\
evaluate (Literal i) \ \ \ \ \ \ \ = i \\
evaluate (Plus expr0 expr1) = \\}
\usebeamerfont*{font DejaVu Sans} {\color{darkblue} Think of interface and subclasses if you like OOP} \\
\usebeamerfont*{font DejaVu Sans Mono} \color{white} . \\
test0 = evaluate (Literal 1) \\
test1 = evaluate (Plus (Literal 1) (Literal 2)) \\
test2 = evaluate (Plus (Plus (Literal 1) \\
\ \ \ \ \ \ \ \ \ \ \ \ \ \ \ \ \ \ \ \ \ \ \ \ \ \ \ \ \ (Literal 2)) \\
\ \ \ \ \ \ \ \ \ \ \ \ \ \ \ \ \ \ \ \ \ \ \ (Literal 3))
}

\frame{\frametitle{Pattern Matching}
\footnotesize
\usebeamerfont*{font DejaVu Sans Mono}
{\color{white} module Main where \\
.} \\
data Expression = Literal Integer \\
 \ \ \ \ \ \ \ \ \ \ \ \ \ \ \ \ | Plus Expression Expression \\
{\color{white}.} \\
evaluate :: Expression -> Integer \\
evaluate (Literal i) \ \ \ \ \ \ \ = i \\
evaluate (Plus expr0 expr1) = \\
 \ \ evaluate expr0 + evaluate expr1 \\
{\color{white}. \\
test0 = evaluate (Literal 1)} \\
\usebeamerfont*{font DejaVu Sans} {\color{darkblue} Think of dynamic\raisebox{.5mm}{\underline{ }}cast or instanceof with a switch if you like OOP} \\
\usebeamerfont*{font DejaVu Sans Mono} \color{white} test2 = evaluate (Plus (Plus (Literal 1) \\
 \ \ \ \ \ \ \ \ \ \ \ \ \ \ \ \ \ \ \ \ \ \ \ \ \ \ \ \ \ (Literal 2)) \\
 \ \ \ \ \ \ \ \ \ \ \ \ \ \ \ \ \ \ \ \ \ \ \ (Literal 3))
}

\frame{\frametitle{First Expression and evaluate (\jref{https://github.com/godfat/sandbox/blob/master/haskell/fpug/01/00.lhs
}{source})}
\footnotesize
\usebeamerfont*{font DejaVu Sans Mono}
module Main where \\
{\color{white}.} \\
data Expression = Literal Integer \\
 \ \ \ \ \ \ \ \ \ \ \ \ \ \ \ \ | Plus Expression Expression \\
{\color{white}.} \\
evaluate :: Expression -> Integer \\
evaluate (Literal i) \ \ \ \ \ \ \ = i \\
evaluate (Plus expr0 expr1) = \\
 \ \ evaluate expr0 + evaluate expr1 \\
{\color{white}.} \\
test0 = evaluate (Literal 1) \\
test1 = evaluate (Plus (Literal 1) (Literal 2)) \\
test2 = evaluate (Plus (Plus (Literal 1) \\
 \ \ \ \ \ \ \ \ \ \ \ \ \ \ \ \ \ \ \ \ \ \ \ \ \ \ \ \ \ (Literal 2)) \\
 \ \ \ \ \ \ \ \ \ \ \ \ \ \ \ \ \ \ \ \ \ \ \ (Literal 3))
}

\frame{\frametitle{Variable and Environment (\jref{https://github.com/godfat/sandbox/blob/master/haskell/fpug/01/01.lhs
}{source})}
\vskip -2.6mm%
\footnotesize
\usebeamerfont*{font DejaVu Sans Mono}
module Main where \\
{\color{white}.} \\
data Expression = Literal Integer \\
 \ \ \ \ \ \ \ \ \ \ \ \ \ \ \ \ | Plus Expression Expression \\
 \ \ \ \ \ \ \ \ \ \ \ \ \ \ \ \ | Variable String \\
\color{white}. \\
type Environment = [(String, Integer)] \\
\color{white}. \\
evaluate :: Expression -> Environment -> Integer \\
evaluate (Literal i) \ \ \ \ \ \ \ env = i \\
evaluate (Plus expr0 expr1) env = \\
 \ \ evaluate expr0 env + evaluate expr1 env \\
evaluate (Variable name) \ \ \ env = \\
 \ \ case lookup name env of (Just i) -> i
}

\frame{\frametitle{Variable and Environment (\jref{https://github.com/godfat/sandbox/blob/master/haskell/fpug/01/01.lhs
}{source})}
\vskip -2.6mm%
\footnotesize
\usebeamerfont*{font DejaVu Sans Mono}
module Main where \\
{\color{white}.} \\
data Expression = Literal Integer \\
 \ \ \ \ \ \ \ \ \ \ \ \ \ \ \ \ | Plus Expression Expression \\
 \ \ \ \ \ \ \ \ \ \ \ \ \ \ \ \ | Variable String \\
{\color{white}.} \\
type Environment = [(String, Integer)] \\
\color{white}. \\
evaluate :: Expression -> Environment -> Integer \\
evaluate (Literal i) \ \ \ \ \ \ \ env = i \\
evaluate (Plus expr0 expr1) env = \\
 \ \ evaluate expr0 env + evaluate expr1 env \\
evaluate (Variable name) \ \ \ env = \\
 \ \ case lookup name env of (Just i) -> i
}

\frame{\frametitle{Variable and Environment (\jref{https://github.com/godfat/sandbox/blob/master/haskell/fpug/01/01.lhs
}{source})}
\vskip -2.6mm%
\footnotesize
\usebeamerfont*{font DejaVu Sans Mono}
module Main where \\
{\color{white}.} \\
data Expression = Literal Integer \\
 \ \ \ \ \ \ \ \ \ \ \ \ \ \ \ \ | Plus Expression Expression \\
 \ \ \ \ \ \ \ \ \ \ \ \ \ \ \ \ | Variable String \\
{\color{white}.} \\
type Environment = [(String, Integer)] \\
{\color{white}.} \\
evaluate :: Expression -> Environment -> Integer \\
\color{white} evaluate (Literal i) \ \ \ \ \ \ \ env = i \\
evaluate (Plus expr0 expr1) env = \\
 \ \ evaluate expr0 env + evaluate expr1 env \\
evaluate (Variable name) \ \ \ env = \\
 \ \ case lookup name env of (Just i) -> i
}

\frame{\frametitle{Variable and Environment (\jref{https://github.com/godfat/sandbox/blob/master/haskell/fpug/01/01.lhs
}{source})}
\vskip -2.6mm%
\footnotesize
\usebeamerfont*{font DejaVu Sans Mono}
module Main where \\
{\color{white}.} \\
data Expression = Literal Integer \\
 \ \ \ \ \ \ \ \ \ \ \ \ \ \ \ \ | Plus Expression Expression \\
 \ \ \ \ \ \ \ \ \ \ \ \ \ \ \ \ | Variable String \\
{\color{white}.} \\
type Environment = [(String, Integer)] \\
{\color{white}.} \\
evaluate :: Expression -> Environment -> Integer \\
evaluate (Literal i) \ \ \ \ \ \ \ env = i \\
evaluate (Plus expr0 expr1) env = \\
 \ \ evaluate expr0 env + evaluate expr1 env \\
evaluate (Variable name) \ \ \ env = \\
 \ \ case lookup name env of (Just i) -> i
}

\frame{\frametitle{Variable and Environment (\jref{https://github.com/godfat/sandbox/blob/master/haskell/fpug/01/01.lhs
}{source})}
\vskip -2.6mm%
\footnotesize
\usebeamerfont*{font DejaVu Sans Mono}
test0 = evaluate (Variable "var") [("var", 1)] \\
test1 = evaluate (Plus (Variable "var") (Literal 2)) [("var", 1)]
}

\frame{\frametitle{Type Alias}
\vskip -2.6mm%
\footnotesize
\usebeamerfont*{font DejaVu Sans Mono}
module Main where \\
{\color{white}.} \\
data Expression = Literal Integer \\
 \ \ \ \ \ \ \ \ \ \ \ \ \ \ \ \ | Plus Expression Expression \\
 \ \ \ \ \ \ \ \ \ \ \ \ \ \ \ \ | Variable {\color{darkblue}Name} \\
{\color{white}.} \\
type {\color{darkblue}Name} = String \\
type Environment = [({\color{darkblue}Name}, Integer)] \\
{\color{white}.} \\
evaluate :: Expression -> Environment -> Integer \\
evaluate (Literal i) \ \ \ \ \ \ \ env = i \\
evaluate (Plus expr0 expr1) env = \\
 \ \ evaluate expr0 env + evaluate expr1 env \\
evaluate (Variable name) \ \ \ env = \\
 \ \ case lookup name env of (Just i) -> i
}
\frame{\frametitle{Pair}
\vskip -2.6mm%
\footnotesize
\usebeamerfont*{font DejaVu Sans Mono}
("var", 2) :: (String, Integer) \\
(2, "var") :: (Integer, String)
}

\frame{\frametitle{List}
\vskip -2.6mm%
\footnotesize
\usebeamerfont*{font DejaVu Sans Mono}
$\begin{array}{l}
\text{[1,2,3] :: [Integer]} \\
\text{["a","b","c"] :: [String]} \\
\text{[("var", 1)] :: [(String, Integer)]}
\end{array}$
}
