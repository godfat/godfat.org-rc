
\section{Let's Implement λ-Calculus}
\subsection{}

\frame{\frametitle{\color{white} dummy}
  \begin{itemize}
    \item {\color{lightgray} What can Haskell do?}
    \item {\color{lightgray} What is λ-Calculus?}
    \item {\color{lightgray} Why Implement λ-Calculus?}
    \item {Let's Implement λ-Calculus}
    \item {\color{lightgray} Questions?}
    \item {\color{lightgray} References}
  \end{itemize}
}

\frame{\frametitle{First Expression and evaluate}
\footnotesize
\usebeamerfont*{font DejaVu Sans Mono}
module Main where \\
\color{white}. \\
data Expression = Literal Integer \\
 \ \ \ \ \ \ \ \ \ \ \ \ \ \ \ \ | Plus Expression Expression \\
{\color{white}.} \\
evaluate :: Expression -> Integer \\
evaluate (Literal i) \ \ \ \ \ \ \ = i \\
evaluate (Plus expr0 expr1) = \\
 \ \ evaluate expr0 + evaluate expr1 \\
{\color{white}.} \\
test0 = evaluate (Literal 1) \\
test1 = evaluate (Plus (Literal 1) (Literal 2)) \\
test2 = evaluate (Plus (Plus (Literal 1) \\
 \ \ \ \ \ \ \ \ \ \ \ \ \ \ \ \ \ \ \ \ \ \ \ \ \ \ \ \ \ (Literal 2)) \\
 \ \ \ \ \ \ \ \ \ \ \ \ \ \ \ \ \ \ \ \ \ \ \ (Literal 3))
}

\frame{\frametitle{First Expression and evaluate}
\footnotesize
\usebeamerfont*{font DejaVu Sans Mono}
module Main where \\
{\color{white}.} \\
data Expression = Literal Integer \\
 \ \ \ \ \ \ \ \ \ \ \ \ \ \ \ \ | Plus Expression Expression \\
\color{white}. \\
evaluate :: Expression -> Integer \\
evaluate (Literal i) \ \ \ \ \ \ \ = i \\
evaluate (Plus expr0 expr1) = \\
 \ \ evaluate expr0 + evaluate expr1 \\
{\color{white}.} \\
test0 = evaluate (Literal 1) \\
test1 = evaluate (Plus (Literal 1) (Literal 2)) \\
test2 = evaluate (Plus (Plus (Literal 1) \\
 \ \ \ \ \ \ \ \ \ \ \ \ \ \ \ \ \ \ \ \ \ \ \ \ \ \ \ \ \ (Literal 2)) \\
 \ \ \ \ \ \ \ \ \ \ \ \ \ \ \ \ \ \ \ \ \ \ \ (Literal 3))
}

\frame{\frametitle{First Expression and evaluate}
\footnotesize
\usebeamerfont*{font DejaVu Sans Mono}
module Main where \\
{\color{white}.} \\
data Expression = Literal Integer \\
 \ \ \ \ \ \ \ \ \ \ \ \ \ \ \ \ | Plus Expression Expression \\
{\color{white}.} \\
evaluate :: Expression -> Integer \\
evaluate (Literal i) \ \ \ \ \ \ \ = i \\
evaluate (Plus expr0 expr1) = \\
 \ \ evaluate expr0 + evaluate expr1 \\
\color{white}. \\
test0 = evaluate (Literal 1) \\
test1 = evaluate (Plus (Literal 1) (Literal 2)) \\
test2 = evaluate (Plus (Plus (Literal 1) \\
 \ \ \ \ \ \ \ \ \ \ \ \ \ \ \ \ \ \ \ \ \ \ \ \ \ \ \ \ \ (Literal 2)) \\
 \ \ \ \ \ \ \ \ \ \ \ \ \ \ \ \ \ \ \ \ \ \ \ (Literal 3))
}

\frame{\frametitle{First Expression and evaluate}
\footnotesize
\usebeamerfont*{font DejaVu Sans Mono}
module Main where \\
{\color{white}.} \\
data Expression = Literal Integer \\
 \ \ \ \ \ \ \ \ \ \ \ \ \ \ \ \ | Plus Expression Expression \\
{\color{white}.} \\
evaluate :: Expression -> Integer \\
evaluate (Literal i) \ \ \ \ \ \ \ = i \\
evaluate (Plus expr0 expr1) = \\
 \ \ evaluate expr0 + evaluate expr1 \\
{\color{white}.} \\
test0 = evaluate (Literal 1) \\
test1 = evaluate (Plus (Literal 1) (Literal 2)) \\
test2 = evaluate (Plus (Plus (Literal 1) \\
 \ \ \ \ \ \ \ \ \ \ \ \ \ \ \ \ \ \ \ \ \ \ \ \ \ \ \ \ \ (Literal 2)) \\
 \ \ \ \ \ \ \ \ \ \ \ \ \ \ \ \ \ \ \ \ \ \ \ (Literal 3))
}

\frame{\frametitle{Algebraic Datatypes}
\footnotesize
\usebeamerfont*{font DejaVu Sans Mono}
{\color{white} module Main where \\
.} \\
data Expression = Literal Integer \\
 \ \ \ \ \ \ \ \ \ \ \ \ \ \ \ \ | Plus Expression Expression \\
{\color{white}. \\
evaluate :: Expression -> Integer \\
evaluate (Literal i) \ \ \ \ \ \ \ = i \\
evaluate (Plus expr0 expr1) = \\}
\usebeamerfont*{font DejaVu Sans} {\color{darkblue} Think of interface and subclasses if you like OOP} \\
\usebeamerfont*{font DejaVu Sans Mono} \color{white} . \\
test0 = evaluate (Literal 1) \\
test1 = evaluate (Plus (Literal 1) (Literal 2)) \\
test2 = evaluate (Plus (Plus (Literal 1) \\
\ \ \ \ \ \ \ \ \ \ \ \ \ \ \ \ \ \ \ \ \ \ \ \ \ \ \ \ \ (Literal 2)) \\
\ \ \ \ \ \ \ \ \ \ \ \ \ \ \ \ \ \ \ \ \ \ \ (Literal 3))
}

\frame{\frametitle{Pattern Matching}
\footnotesize
\usebeamerfont*{font DejaVu Sans Mono}
{\color{white} module Main where \\
.} \\
data Expression = Literal Integer \\
 \ \ \ \ \ \ \ \ \ \ \ \ \ \ \ \ | Plus Expression Expression \\
{\color{white}.} \\
evaluate :: Expression -> Integer \\
evaluate (Literal i) \ \ \ \ \ \ \ = i \\
evaluate (Plus expr0 expr1) = \\
 \ \ evaluate expr0 + evaluate expr1 \\
{\color{white}. \\
test0 = evaluate (Literal 1)} \\
\usebeamerfont*{font DejaVu Sans} {\color{darkblue} Think of dyanamic\raisebox{.5mm}{\underline{ }}cast or instanceof with a switch if you like OOP} \\
\usebeamerfont*{font DejaVu Sans Mono} \color{white} test2 = evaluate (Plus (Plus (Literal 1) \\
 \ \ \ \ \ \ \ \ \ \ \ \ \ \ \ \ \ \ \ \ \ \ \ \ \ \ \ \ \ (Literal 2)) \\
 \ \ \ \ \ \ \ \ \ \ \ \ \ \ \ \ \ \ \ \ \ \ \ (Literal 3))
}
